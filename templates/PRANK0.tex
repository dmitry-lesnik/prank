\documentclass[10pt]{article}
%To use pdflatex, uncomment these lines, as well as the \href lines
%in each entry
%\usepackage[pdftex,
%       colorlinks=true,
%       urlcolor=blue,        \href{...}{...} external (URL)
%       filecolor=green,      \href{...} local file
%       linkcolor=red,        \ref{...} and \pageref{...}
%       pdftitle={Papers by AUTHOR},
%       pdfauthor={Your Name},
%       pdfsubject={Just a test},
%       pdfkeywords={test testing testable},
%       pagebackref,
%       pdfpagemode=None,
%        bookmarksopen=true]{hyperref}
\usepackage{color}
\usepackage[dvips]{graphics}
\renewcommand{\familydefault}{\sfdefault} % San serif
\renewcommand{\labelenumii}{\arabic{enumi}.\arabic{enumii}}

\pagestyle{empty}
\oddsidemargin 0.0in
\textwidth 6.5in
\topmargin -0.35in
\textheight 10.0in

\begin{document}


\title{Difference between WKB and time-domain integration methods}
\date{}
\maketitle



We shall consider spherically symmetric metrics which can be written (in $3+1$- space-time dimensions) in the following form

\begin{equation}\label{MT}
ds^2 = A(r) dt^2 + B(r) d r^2 + r^2 (d \theta^2 + \sin^2 \theta d \phi^2).
\end{equation}

In higher than $D>4$ dimensions, the 2-sphere is replaced by a $D-2$-sphere. Then, we shall consider fields of various spin (scalar ($s=0$), electromagnetic ($s=1$), Dirac ($s=\pm1/2$)).

\begin{equation}\label{wavelike}
\frac{d^{2} \Psi_{i}}{d r_{*}^{2}} + (\omega^{2} - V_{i}(r))\Psi_{i} =
0,\qquad d r_{*}= (A(r) B(r))^{-1/2} dr,
\end{equation}

The "tortoise" coordinate $r^*$ spans from $- \infty$ (black hole horizon) to $+ \infty$ (spacial infinity). Sometimes the space cannot be extended to infinity but to some other, distant cosmological horizon. The effective potential usually has the form of a smooth potential barrier with a peak at about 1.5 of the event horizon (from the origin).\\ 

Some facts about effective potential and WKB:\\

1) At the far asymptotic (infinity or cosmological horizon) the effective potential must approach some finite constant values. If not so, the program should cancel computing. Usually for astrophysically reasonable setup the fall-off of an effective potential far from the black hole is required.\\
    
2) If the effective potential is positive definite everywhere outside the black hole (i.e. from the event horizon until infinity), the perturbation of the field under consideration is stable.\\

3) If the effective potential has some region where it is negative, the stability is not guaranteed. \\

4) If there are negative gaps and, thereby, more than 2 turning points, the WKB approach in its traditional form (called Will-Schutz form, the one we are using) cannot be effectively applied, because in the Will-Schutz formula takes into consideration only the major contribution of scattering only around the main peak of the potential barrier. Thus, usually WKB is bad for catching an instability.\\

5) The WKB method is good for higher multipole numbers $\ell$. Remember, $\ell (\ell +1)/r^2$ is the centrifugal barrier, like in a quantum mechanical problem for a hydrogen atom? In general WKB is valid until $\ell \geq n$, where $n$ is the overtone number. The smaller $\ell$ and the higher $n$, the worse is the accuracy of WKB formula.

6) WKB series converges only asymptotically, i.e. it does not guarantee accuracy for each case, but, it is economic and can be easily automatized.\\ 


Some facts about time-domain integration\\

1) The profile of a signal can in principle be obtained with any desirable accuracy and in most cases the dominant frequency ($n=0$ called the fundamental mode) can be extracted with any desirable accuracy. \\

2) In time-domain integration we have derivatives in time direction instead of $\omega$, so that contribution of all modes are included. Therefore, time-domain integration allows one to detect the instability (if any).\\

3) Time domain integration for lengthy potentials may be much more expensive in terms of computer time than the WKB.\\


Thus\\

\textbf{Case 1.} We plot the effective potential and find that it has more than one local extremum. Then we say:\\

"In addition to the main peak, the effective potential has other extremum (extremae). Therefore, WKB method [reference] cannot be accurate for all values of the parameters as it takes into consideration only scattering around the main peak."\\

\textbf{Case 2.} We plot the effective potential and it shows no peak at all for some values of parameters. Then we say:\\

"For THESE values of parameters the effective potential has no peak, and therefore the WKB approach cannot be applied."\\

In this case we simply has no data in the table against WKB column.\\

\textbf{Case 3.} We plot the effective potential and it has normal behavior - effective potential has the form of a smooth potential barrier with one peak and falling-off at both infinities. Yet, at the same time the results for $\omega$ differ. Then we have a few sub-cases: \\

3a) The difference is relatively small, from a fraction of one percent to a 5-6 or even 10 percents. Then we can say:\\

"The results obtained by WKB method is in concordance with the ones obtained by time-domain integration. The relatively small difference can be explained by not very high accuracy of the WKB method for the lowest multipole $\ell$ considered here. "\\

3b) For some values of parameters the difference is big ( tenths of percents). The, we say:\\

"For THESE values of parameters we see that the results obtained with the WKB method considerably differs from the ones obtained by time-domain integration. As in general case, WKB series converges only asymptotically here we should rely upon the time-domain integration, which shows convergence."\\

One should remember that for different values of parameters there may be different cases -types of behavior.\\

Now i will summarize possible (but not highly probable) problems with \textbf{the time-domain integration}.\\

\textbf{Case A}. We have a frequency in the table and it is stable - everything is OK. We say "The time-domain integration take into account contribution of all modes into the signal, so that the damping profiles prove stability of THESE field at THESE values of parameters."\\

\textbf{Case B.} We have a frequency in the table, but it is unstable, then the program should say "The time-domain integration shows unbounded growth of the perturbation which means instability for THESE values of parameters." \\

\textbf{Case C.} On plot we can see the damping profile, but we cannot extract a dominant frequency with sufficient accuracy (no data for $\omega$ in the table), then we say " Time-domain integration produced such a profile of a signal that does not allow one to extract the frequency of the fundamental mode by the Prony method. Nevertheless, we see that the field is stable at THESE values of parameters. " (no matter what is this "Prony".)\\


If we find no instability in the paper, we say in the conclusion:\\

Case $\alpha$ :  "We have shown that the THIS (scalar or Dirac or Maxwell or some of them or all of them) field(s) is (are) stable in the background of OUR black hole. However, this does not guarantee the stability of the background itself under gravitational perturbations. The stability against gravitational perturbations should be tested in a separate work."\\

If we found instability for some (or all) values of parameters, we say: \\

Case $\beta$ : " We have shown that THIS (scalar or Dirac or Maxwell or some of them or all of them) field(s) is unstable in the background of OUR black hole. Such an instability of test scalar field does not mean the black hole instability and simply means that the considered field should posses also the corresponding quantum instability. Nevertheless, this makes appealing investigation of gravitational perturbations of OUR black hole as well, in order to test stability of the background itself."\\


The phrases here will be certainly written better in the final version and with a number of variants.




 
 


\end{document} 